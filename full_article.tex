\documentclass[num-refs]{wiley-article}
\usepackage{graphicx}
\usepackage[space]{grffile}
\usepackage{latexsym}
\usepackage{textcomp}
\usepackage{longtable}
\usepackage{tabulary}
\usepackage{booktabs}
\usepackage{amsfonts,amsmath,amssymb}
\usepackage{hyperref}
\hypersetup{colorlinks=false,pdfborder={0 0 0}}
\usepackage{etoolbox}
\usepackage{placeins}
\usepackage[%
    inline = true,%
    margin = false,%
    status = draft,%
    % author = EJB%
]{fixme}
\fxusetheme{colorsig}
% \FXRegisterAuthor{kdj}{akdj}{KDJ}
\makeatletter
%\patchcmd\@combinedblfloats{\box\@outputbox}{\unvbox\@outputbox}{}{%
%  \errmessage{\noexpand\@combinedblfloats could not be patched}%
%}%
\makeatother
% You can conditionalize code for latexml or normal latex using this.
\newif\iflatexml\latexmlfalse
\AtBeginDocument{\DeclareGraphicsExtensions{.pdf,.PDF,.eps,.EPS,.png,.PNG,.tif,.TIF,.jpg,.JPG,.jpeg,.JPEG}}

\usepackage[utf8]{inputenc}
\usepackage[english]{babel}

% Add any additional LaTeX packages and macros here
\usepackage{microtype}
% \usepackage{siunitx}

%---

\papertype{Software News and Updates}

\title{cclib 2.0, much wow}

\author[1]{Karol M. Langner}
\author[2,3]{Eric J. Berquist}
\author[4]{Noel M. O'Boyle}
\author[5]{Adam L. Tenderholt}
\author[6]{Kunal Sharma}
\author[7]{Sanjeed Schamnad}
\author[3,8]{Shiv Upadhyay}
\author[9]{Minsik Cho}
\author[10]{Jonathon Vandezande}
\author[11]{Sagar Gaur}
\author[12]{Maxim Stolyarchuk}
\author[13]{Felipe S. S. Schneider}
\author[3,8]{Geoffrey R. Hutchison}
% TODO additional authors
% TODO author order
% TODO get affiliations

\affil[1]{Google, Mountain View, CA}
\affil[2]{Q-Chem Incorporated, 6601 Owens Drive, Suite 105, Pleasanton, CA 94588}
\affil[3]{Department of Chemistry, University of Pittsburgh, Pittsburgh, PA 15260}
\affil[4]{4}
\affil[5]{5}
\affil[6]{6}
\affil[7]{7}
\affil[8]{PQI}
\affil[9]{Department of Chemistry, Brown University, Providence, RI 02912}
\affil[10]{MPI}
\affil[11]{11}
\affil[12]{12}
\affil[13]{Universidade Federal de Santa Catarina, Florian\'{o}polis, Brazil}

\runningauthor{Authors}

%---

\iflatexml
% Add any LateXML specific commands here

%---

\else
% The commands below will only change the exported PDF. Edit or remove as needed

\paperfield{Field of the paper}
% \abbrevs{ABC, a black cat; DEF, doesn't ever fret; GHI, goes home immediately.}
\corraddress{Author One PhD, Department, Institution, City, State or Province, Postal Code, Country}
\corremail{correspondingauthor@email.com}
\presentadd{Department, Institution, City, State or Province, Postal Code, Country}
% \fundinginfo{Funder One, Funder One Department, Grant/Award Number: 123456, 123457 and 123458; Funder Two, Funder Two Department, Grant/Award Number: 123459}
\fi
%---

\begin{document}
\maketitle
\selectlanguage{english}
\begin{abstract}
This is a LaTeX template designed for use by the \textbf{International Journal of Quantum Chemistry}. Please consult the journal's author guidelines in order to confirm that your manuscript complies with the journal's requirements. When you are ready to submit your manuscript, download/export it in LaTeX or Word format and submit your document at \href{https://mc.manuscriptcentral.com/qua}{https://mc.manuscriptcentral.com/qua}. Please replace this text with your abstract.

\textbf{Keywords} --- keyword 1, \emph{keyword 2}, keyword 3, keyword 4, keyword 5, keyword 6, keyword 7.
\end{abstract}

\section{Introduction}

cclib\cite{Oboyle2008} (\url{http://cclib.github.io/} and \url{https://github.com/cclib/cclib}) is an open-source library, written in Python, for parsing and interpreting the results of computational chemistry packages.  The goals of cclib are centered around the reuse of data obtained from these programs and contained in output files, specifically:
\begin{itemize}
\item extract (parse) data from the output files generated by multiple programs,
\item provide a consistent interface to the results of computational chemistry calculations, particularly those results that are useful for algorithms or visualization,
\item facilitate the implementation of algorithms that are not specific to a particular computational chemistry package, and
\item to maximize interoperability with other open source computational chemistry and cheminformatic software libraries.
\end{itemize}

on Zenodo since version 1.2\cite{Langner2014}, with the latest version being 1.7\cite{Berquist2021}

\section{Open development workflow and the cclib community}

TODO say something about Open Chemistry here?

\subsection{Contributing}

The primary driving force behind each of the author's individual contributions has been the need

\subsection{Google Summer of Code}

Since 2015 (TODO), cclib has participated in Google Summer of Code (GSoC, \url{https://summerofcode.withgoogle.com/}), held between TODO and TODO each year.  Rather than be an individual organization, we apply under the Open Chemistry umbrella group, alongside Avogadro, Open Babel, DeepChem, RDKit, gnina, 3Dmol.js, NWChem, and Psi4.  Open Chemistry maintains a yearly list of project ideas, the latest iteration of which is located at \url{https://wiki.openchemistry.org/GSoC_Ideas_2020}.  cclib has gained a number of sizable contributions through GSoC which would have been time-consuming or infeasible with only the core team.

TODO put in paragraph form:
\begin{itemize}
\item 2015: ?
\item 2016: ?
\item 2017: ?
\item 2018: ?
\item 2019: did not participate
\item 2020: atomic partial charge methods (Minsik)
\end{itemize}

\section{Using cclib with the CLI/API interfaces}

\section{Highlighted developments in cclib 2.0}
\subsection{Parsers}
\subsection{Bridges}
\subsection{Methods}

\section{The future of cclib}

Despite the seemingly outdated idea of parsing human-readable text into machine-usable data,

large-scale workflows.\cite{Abbott2019,StJohn2020}

TODO insert bar plot of citation counts of original paper for each year since publication

there is still a long tail of computation that is performed outside of JSON

the computational chemistry equivalent of the Protein Data Bank does not exist

like Open Babel and RDKit, which have organically grown into toolboxes covering large parts of cheminformatics space. Through its Python interface, RDKit in particular has its force fields integrated as part of QCEngine and is therefore generally available throughout the MolSSI ecosystem.

% \subsection{issues}
% \subsection{99  Future of nmo, nbasis and similar attributes feature, question}
% \subsection{131 Support fragment calculations in some way feature, parsers}
% \subsection{132 Support ONIOM calcs in some way Gaussian, feature}
% \subsection{180 Support Cfour output feature, parsers}
% \subsection{241 Strip output file timings Gaussian, beginner, question}
% \subsection{254 Parse GAMESS *.dat files GAMESS}
% \subsection{299 Implement gradients and Hessians parsers}
% \subsection{390 New attribute scfinfo for microiteration data. GAMESS, parsers, question}
% \subsection{419 Structure of future attributes feature, parsers, question}
% \subsection{474 Add finite difference gradients/Hessians to unit tests feature, tests}
% \subsection{518 Sunsetting Python2 maintenance, question}
% \subsection{537 Investigate initialization of logfile superclass maintenance, parsers}
% \subsection{612 Consider moving data notes into the code beginner, docs}
% \subsection{628 Rethink output of methods methods, question}
% \subsection{653 Natural Spin Orbitals parsing feature, parsers}
% \subsection{657 Support multistep jobs generally feature, parsers}
% \subsection{776 Replace periodictable and SciPy with QCElemental beginner, feature, maintenance}
% \subsection{890 Standardize Bridge (and methods?) API bridge, methods}
% \subsection{PRs}
% \subsection{781 Replace periodictable and SciPy with QCElemental}
% \subsection{798 QCSchema: add output writer}

\section*{Acknowledgements}
Acknowledgements should include contributions from anyone who does not meet the criteria for authorship (for example, to recognize contributions from people who provided technical help, collation of data, writing assistance, acquisition of funding, or a department chairperson who provided general support), as well as any funding or other support information.

\section*{Conflict of interest}
You may be asked to provide a conflict of interest statement during the submission process. Please check the journal's author guidelines for details on what to include in this section. Please ensure you liaise with all co-authors to confirm agreement with the final statement.
%
\FloatBarrier{}
\bibliography{bibliography.bib}
\end{document}
